\usepackage[pdftex]{color}
\usepackage{fancybox}
\usepackage{fancyvrb}

\newcommand{\htmldiv}[2]{\HTMLcode[id="#1"]{div}{#2}}

%%
%% COLOR
%%

\definecolor{backgroundgray}{gray}{0.9}

%%
%% FANCYVRB
%%

% Las tabulaciones se substituyen por dos espacios
\fvset{tabsize=2}

% Creamos un nuevo environment de fancyvrb para los ejemplos enmarcados
\DefineVerbatimEnvironment{VerbShell}{BVerbatim}{boxwidth=0.8\textwidth,fontsize=\small,samepage=true,commandchars=\\\{\}}
\DefineVerbatimEnvironment{VerbWideShell}{BVerbatim}{boxwidth=0.96\textwidth,fontsize=\small,commandchars=\\\{\}}

\newenvironment{shellverbatim}%
{\VerbatimEnvironment\begin{Sbox}\begin{VerbShell}}%
{\end{VerbShell}\end{Sbox}\setlength{\fboxsep}{8pt}\begin{center}\fcolorbox{black}{backgroundgray}{\TheSbox}\end{center}}

\newenvironment{wideshellverbatim}%
{\VerbatimEnvironment\begin{Sbox}\begin{VerbWideShell}}%
{\end{VerbWideShell}\end{Sbox}\setlength{\fboxsep}{8pt}\begin{center}\fcolorbox{black}{backgroundgray}{\TheSbox}\end{center}}

\newenvironment{warning}
{\latexhtml{\begin{importantnote}}{\begin{rawhtml}<div class='warning'>\end{rawhtml}}}
{\latexhtml{\end{importantnote}}{\begin{rawhtml}</div>\end{rawhtml}}}
