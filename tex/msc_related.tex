Many projects tackle the problem of dynamically overlaying virtual
resources on top of physical resources by using virtualization
technologies, and do so with different resource models. These models
generally consider overhead as part of the virtual resource allocated
to the user, or do not manage or attempt to reduce it. A common
assumption in related projects is that all necessary images are already
deployed on the worker nodes. Our requirements for dynamic deployment
of AR and ASAP workspaces make it impossible to make this assumption.

The Shirako system \cite{BorjaCite12} developed within the Cluster{}-On{}-Demand
project \cite{BorjaCite10, codweb} uses VMs to partition a physical cluster into several
virtual clusters. Their interfaces focus on granting \emph{leases} on
resources to users, which can be redeemed at some point in the future.
However since their model focuses on batch cases the adopted overhead
management model is to absorb it into resources used for VM deployment
and management. As we have shown, this model is not sufficient for
AR{}-style cases.

The XGE project \cite{xge} extends SGE so it will use different VMs for serial batch requests and for parallel job requests. The motivation for their work is improving utilization of a university cluster shared by two user communities with different requirements. By using the suspend/resume capabilities of Xen virtual machines when combining serial and parallel jobs, the XGE project has achieved improved cluster utilization when compared against using backfilling and physical hardware. However, the XGE project assumes two fixed VM images predeployed on all cluster nodes.


The VIOLIN and VioCluster projects \cite{BorjaCite13, viocluster, DBLP:journals/computer/RuthJXG05} allow users to overlay a
virtual cluster over more than one physical cluster, leveraging VM live
migration to perform load balancing between the different clusters. The
VioCluster model assumes that VM images are already deployed on
potential hosts, and only a ``binary diff'' file (implemented as a
small Copy{}-On{}-Write file), expressing the particular configuration
of each instance, is transferred at deploy{}-time. This is less
flexible than using image metadata, as COWs can be invalidated by
changes in the VM images. Furthermore, our work focuses on use cases
where multiple image templates might be used in a physical cluster,
which makes it impractical to prestage all the templates on all the
nodes.

The Maestro{}-VC system \cite{BorjaCite18} also explores the benefits of providing a
scheduler with application{}-specific information that can optimize its
decisions and, in fact, also leverages caches to reduce image
transfers. However, Maestro{}-VC focuses on clusters with long
lifetimes, and their model does not schedule image transfer overhead in
a deadline{}-sensitive manner, and just assumes that any image staging
overhead will be acceptable given the duration of the virtual cluster.
Our work includes short{}-lived workspaces as a case that must perform
efficiently under our model.

The Virtuoso Project \cite{BorjaCite19} and, in particular, its VSched component \cite{BorjaCite17},
is capable of co{}-scheduling both interactive and batch workloads on
individual machines in a deadline{}-sensitive manner, but does not
factor in the overhead of deploying the VMs to the nodes where they are
needed.

The In{}-VIGO project \cite{BorjaCite16} proposes adding three layers of
virtualization over grid resources to enable the creation of virtual
grids. Our work, which relates to their first layer (creating virtual
resources over physical resources), is concerned with finer{}-grained
allocations and enforcements than in the In{}-VIGO project. Although some exploration of cache-based deployment has also been done with VMPlant \cite{BorjaCite20}, this project focuses on batch as opposed to deadline-sensitive cases.


